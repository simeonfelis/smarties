\hypertarget{menu_8h}{
\section{menu.h File Reference}
\label{menu_8h}\index{menu.h@{menu.h}}
}
The menu structure and handling. 



This graph shows which files directly or indirectly include this file:\nopagebreak
\begin{figure}[H]
\begin{center}
\leavevmode
\includegraphics[width=155pt]{menu_8h__dep__incl}
\end{center}
\end{figure}
\subsection*{Data Structures}
\begin{CompactItemize}
\item 
struct \hyperlink{structmenu__entry__t}{menu\_\-entry\_\-t}
\begin{CompactList}\small\item\em The menu structure. \item\end{CompactList}\end{CompactItemize}


\subsection{Detailed Description}
The menu structure and handling. 

Copyright (c) 2008 Simeon Felis\hypertarget{index_intro}{}\subsection{License}\label{index_intro}
GPL2 Licence\hypertarget{index_arch}{}\subsection{Architecture}\label{index_arch}
This is not an up to date, complete reference for the whole menu structure. However this is a rather completet explanation on how the menu works with examples.

The menu-control is locked to a 2-line alphanumeric display. It will be entered as soon as the smartie sorter is in \hyperlink{system_8h_6e4cea4b6246f491c526211f5405627df1c8a041b6d3d46002d02ed0645a8ba2}{SYS\_\-MODE\_\-PAUSE}. Otherwise the lcd displays some status information:

During \hyperlink{system_8h_6e4cea4b6246f491c526211f5405627d6f86e68a36933ceea048d78b95769945}{SYS\_\-MODE\_\-INIT} : 

\begin{Code}\begin{verbatim}  +--------------+
  | INITIALIZING |
  |              |
  +--------------+
\end{verbatim}
\end{Code}



During \hyperlink{system_8h_6e4cea4b6246f491c526211f5405627d564529220e61e848983d8eb9d7f87736}{SYS\_\-MODE\_\-RUNNING} : 

\begin{Code}\begin{verbatim}  +--------------+
  | ENTER PAUSE  |
  |RUN        COL|
  +--------------+
\end{verbatim}
\end{Code}



During \hyperlink{system_8h_6e4cea4b6246f491c526211f5405627df1c8a041b6d3d46002d02ed0645a8ba2}{SYS\_\-MODE\_\-PAUSE} : Greeting menu: 

\begin{Code}\begin{verbatim}   right     +--------------+    right     +--------------+   right
 <---------> |  ENTER MENU  | <----------> |    RESUME    | <--------->
   left      |<prev   next >|    left      |<p          n>|   left
             +--------------+              +--------------+
               |                             |
               |enter_submenu()              |sys_resume()
\end{verbatim}
\end{Code}



MAIN menu: 

\begin{Code}\begin{verbatim}  r   +--------------+   r   +--------------+   r   +--------------+   r   +--------------+   r
<---> | ROTATE REV   | <---> | ROTATE CATCH | <---> | TCS colors   | <---> |   Go Back    | <--->
  l   |<p          n>|   l   |<p          n>|   l   |<p          n>|   l   |<p          n>|
      +--------------+       +--------------+       +--------------+       +--------------+
      |                       |                       |                      |
      |sys_rotate_revolver()  |sys_rotate_catcher()   |sys_tcs_colors()      |enter_topmenu()
\end{verbatim}
\end{Code}



Each menu layer has its own array. Example:

MAIN menu 

\begin{Code}\begin{verbatim}+--------------+  +--------------+  +--------------+ +--------------+
|      [0]     |  |      [1]     |  |      [2]     | |      [3]     |
|              |  |              |  |              | |              |
+--------------+  +--------------+  +--------------+ +--------------+
\end{verbatim}
\end{Code}



And so on.

The Display has two lines with 24 characters. Here the exact layout during SYS\_\-MODE\_\-RUNNING: 

\begin{Code}\begin{verbatim}      1  2  3  4  5  6  7  8  9 10 11 12 13 14 15 16 17 18 19 20 21 22 23 24
    + -  -  -  -  -  -  -  -  -  -  -  -  -  -  -  -  -  -  -  -  -  -  -  - +
  1 |                            T  I  T  L  E                               |
  2 | S  T  A  T  : [   M  O  D  E      ]       C  O  L  :[ C  O  L  O  R   ]|
    + -  -  -  -  -  -  -  -  -  -  -  -  -  -  -  -  -  -  -  -  -  -  -  - +
\end{verbatim}
\end{Code}



\begin{itemize}
\item Line 1: Column 1 to 24 is reserved for the title. If the push button is pressed, the action described by the title will be executed.\item Line 2:\begin{itemize}
\item Column 6 to 12 is reserved for the current mode. Following modes can be displayed:\begin{itemize}
\item PAUSE\item RUNNING\end{itemize}
\item Column 19 to 24 is reserved for the folling colors:\begin{itemize}
\item yellow: YELLOW\item red: RED\item blue: BLUE\item brown: BROWN\item green: GREEN\item purple: PURPLE\item unknown: UNKNOWN\end{itemize}
\end{itemize}
\end{itemize}


Note: the column numbers may be out of date.

For some setting possibilties have a look at \hyperlink{lcd__display_8h}{lcd\_\-display.h}

For lcd control functions have a look at \hyperlink{lcd__display_8c}{lcd\_\-display.c}

The init of the menu is done at \hyperlink{inits_8c_5cf20cf8f8b8d7c2aeaaea014d157583}{init\_\-menu()} in \hyperlink{inits_8c}{inits.c} 

Definition in file \hyperlink{menu_8h-source}{menu.h}.